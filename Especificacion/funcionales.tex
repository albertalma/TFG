\subsection{Requisitos funcionales}
Dado que el proyecto utiliza la metodologia agil \textit{SCRUM} los requisitos funcionales de este vienen definidos por historias de usuario, las cuales se encargan de describir las distintas funcionalidades del sistema.
\subsubsection{Historias de usuario}

Las historias de usuario que se enumeran a continuación se han ido definiendo a lo largo del proyecto durante los \textit{Sprint Plannings}.

Las partes que contiene cada historia de usuario son:
\begin{description}
\item[Título] Identifica la historia de usuario (contiene un número pero simplemente para tener las distintas historias enumeradas).
\item[Texto] Se especifica \textbf{quien} quiere realizar dicha historia, \textbf{que} quiere realizar y \textbf{para que}.
\item[Criterios de aceptación] Una lista con los criterios que ha de cumplir dicha funcionalidad para que sea aceptada por negocio.
\end{description}

La historia de usuario 0, nos muestra un ejemplo de la estructura de estas:


%Macro para pintar historias de usuario
\newcounter{counterHistorias} 



\newcommand{\historiaDeUsuario}[5]{
\begin{table}[h]
\begin{center}
\begin{tabular}{@{}l@{}}
\toprule
\large{\textbf{\arabic{counterHistorias} {#1}}}   
\\ \midrule
\begin{minipage}{5in}
            \vskip 4pt
            \begin{description}
                \item[Como] {#2}
                \item[Quiero] {#3}
                \item[Para poder] {#4}
            \end{description}
            \vskip 4pt
        \end{minipage}           
\\ \midrule
Criterios de aceptación 
\\ \midrule
\begin{minipage}{5in}
            \vskip 4pt
            \begin{itemize}
                {#5}
            \end{itemize}
            \vskip 4pt
        \end{minipage}                 
\\ \bottomrule
\end{tabular}
\end{center}
\end{table}
\stepcounter{counterHistorias}
}

    
\historiaDeUsuario
{Historia de Usuario}
{Usuario}
{hacer algo}
{obtener un resultado}
{
    \item Criterio 1.
    \item Criterio 2.
}

\historiaDeUsuario
{Registrarse en el sistema}
{Usuario}
{poder darme de alta en el sistema}
{posteriormente entrar en él}
{
\item Mostrar un formulario de registro compuesto por: e-mail (único) y contraseña.
\item Una vez rellenado el formulario guardar los datos del usuario.
}

\historiaDeUsuario
{Iniciar sesión}
{Usuario}
{acceder al sistema}
{acceder a las funcionalidades}
{
\item Mostrar un formulario de inicio de sesión compuesto por: e-mail y contraseña.
    \item Entrar en el sistema con las credenciales correctas.
    \item Mostrar perfil del usuario.
    \item No se ha de poder entrar sin unas credenciales válidas.
}

\historiaDeUsuario
{Cerrar sesión}
{Usuario}
{cerrar la sesión actual}
{cambiar de usuario}
{
    \item En la barra del portal aparecerá la opción de cerrar sesión
    \item Una vez se haya cerrado la sesión, se mostrará la pantalla de inicio de sesión como muestra de que no hay ninguna sesión abierta.
}

\historiaDeUsuario
{Visualizar perfil}
{Usuario}
{visualizar mi perfil}
{ver y editar mi información y la de mis edificios}
{
    \item Mostrar perfil dependiendo del tipo de usuario.
    \item \textbf{Empresa} se trata del caso concreto de VIAS y LAVOLA, se les mostrara una lista con los edificios que tienen a cargo .
    \item \textbf{Terciario} usuarios que cuenten con edificios del sector terciario (administración, negocios...), se les mostrará un perfil más orientado a sus necesidades.
    \item \textbf{Residencial} usuarios que cuenten con hogares particulares, se les mostrara toda la parte de gamificación.
}

\historiaDeUsuario
{Perfil empresa}
{Empresa}
{ver una lista de mis edificios}
{ver y editar la información de estos}
{
    \item Mostrar una lista con los edificios que ha creado la empresa
    \item Acceder al formulario de creación de edificios.
    \item Acceder a la edición de los edificios.
    \item Eliminar edificios.
}

\historiaDeUsuario
{Definir edificio}
{Usuario}
{crear un edificio}
{guardar su información en el sistema y posteriormente acceder a más funcionalidades}
{
    \item El Formulario tendrá que contar con los siguientes campos para definir el edificio: provincia (a escoger entre Madrid o Barcelona), población, tipo de calle, dirección y número. 
    \item Definir el número de bloques, escaleras, plantas y puertas.
    \item Definir el sector del edificio.
}

\historiaDeUsuario
{Definir Factores Pasivos}
{Usuario}
{definir los factores pasivos de mi edificio}
{guardar la información para las posteriores recomendaciones}
{
    \item Definir solución constructiva del edificio entre: M1, M2, M3, M4(U) Unifamiliar, M4(P) Plurifamiliar o M5.
    \item Definir la cubierta del edificio entre: C1 Inclinada, C2 Plana, C3 Inclinada, C4 Plana o Ninguna.
    \item Definir acristalamiento entre: H1 Simple, H2 Simple, H3 Doble.
    \item Especificar año de construcción y de la última remodelación.
}

\historiaDeUsuario
{Definir Factores Activos I}
{Usuario}
{definir los factores activos de mi edificio (ACS, refrigeración y calefacción)}
{guardar la información para las posteriores recomendaciones}
{
    \item Definir el Agua Caliente y Sanitaria (ACS) que utiliza el edificio: Ninguno, Termoacumulador electrico, Termoacumulador gas, Caldera Gasoil. Caldera GLP o Caldera Gas Natural.
    \item Definir el sistema de refrigeración: Ninguno, Bomba de calor o Planta enfriadora.
    \item Definir el sistema de calefacción: Ninguno, Bomba de Calor, Caldera Gasoil. Caldera GLP o Caldera Gas Natural.
}

\historiaDeUsuario
{Definir Factores Activos II}
{Usuario}
{definir los factores activos de mi edificio (Equipos)}
{guardar la información para las posteriores recomendaciones}
{
    \item Dependiendo del sector al que pertenezca el edificio mostrar una lista con los diferentes equipos que puede contener.
    \item Poder escoger que equipos contiene el edificio y cantidad.
}

\historiaDeUsuario
{Definir Factores Activos III}
{Usuario}
{definir los factores activos de mi edificio (Iluminación)}
{guardar la información para las posteriores recomendaciones}
{
    \item Mostrar una lista con los distintos tipos de iluminación que puede tener el edificio.
    \item Escoger los distintos tipos de iluminación y cantidad de cada una.
}

\historiaDeUsuario
{Obtener información del Catastro}
{Usuario}
{obtener información del catastro}
{tener una información fiable y no tener que rellenar todos los datos del formulario}
{
    \item A partir de los datos del edificio conectar con el servicio del Catastro.
    \item Una vez obtenidos los datos, parsearlos para poder obtener aquellos que son necesarios.
    \item Mostrar en el formulario los datos obtenidos del catastro.
}

\historiaDeUsuario
{Editar Edificio}
{Empresa}
{editar los datos de mis edificios}
{modificar los datos ya sea, por reformas o por algún error en los datos}
{
    \item Mostrar los distintos hogares asociados al edificio.
    \item Poder editar los factores activos y pasivos como en las historias de usuario de definición de factores activos y pasivos. 
}

\historiaDeUsuario
{Editar hogares asociados a edificios}
{Empresa}
{editar unos datos concretos de los hogares que están asociados a mis edificios}
{modificar estos datos para que no sea necesario que el usuario particular tenga esa información}
{
    \item Definir la orientación del hogar: Norte, Noreste, Este, Sudeste, Sur, Sudoeste, Oeste y Noroeste.
    \item Definir tipología, la lista de tipologías dependerá del sector del edificio.
    \item Definir superficie del edificio.
}

\historiaDeUsuario
{Promocionar Hogar}
{Empresa}
{asignar hogares a usuarios particulares}
{asignar los hogares de mis edificios a los usuarios que considere oportunos}
{
    \item Mostrar una lista con los usuarios registrados en el sistema.
    \item Asignar un hogar al usuario seleccionado.
}

\clearpage

    
    
