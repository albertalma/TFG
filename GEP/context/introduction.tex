\subsection{Contexto}
Como ya hemos dicho la finalidad de este proyecto será la realización de una aplicación gamificada para incidir en el comportamiento de los usuarios para mejorar la eficiencia energética de sus hogares, es decir un sistema de gestión energética.

Antes de seguir, convendría explicar un poco más en detalle algunos conceptos:

\subsubsection{Sistema de gestión energética}
Un sistema de gestión energética es el conjunto de elementos de una organización dedicados a desarrollar e implantar políticas energéticas, así como gestionar todos los elementos que interactúan con el uso de la energía (aspectos energéticos).

Además este asegura el control y seguimiento de los aspectos energéticos. Esto contribuye a un uso más eficiente y sostenible de la energía.
\subsubsection{Gamificación}
Gamificación~\footnote{Descripción obtenida de http://www.gamificacion.com/que-es-la-gamificacion} es el empleo de mecánicas de juego en entornos y aplicaciones no lúdicas con el fin de potenciar la motivación, la concentración, el esfuerzo, la fidelización y otros valores positivos comunes a todos los juegos. Se trata de una nueva y poderosa estrategia para influir y motivar a grupos de personas.

\subsubsection{Partes Interesadas}
\paragraph{Lavola S.A y Vías y Construcciones S.A}
Lavola S.A~\cite{lavola} (LAVOLA) y Vías y Construcciones S.A~\cite{vias} (VIAS) tienen el rol de clientes dentro del proyecto.

Este proyecto establece una alianza entre el sector de la sostenibilidad, representado por LAVOLA y el sector de la construcción, representado por VÍAS, ambos sectores complementarios y que aportan dos puntos de vista muy distintos a este proyecto lo cual proporcionará una dimensión muy equilibrada a este.

Por tanto al tratarse de los clientes su objetivo será obtener una herramienta interactiva capaz de incidir en los patrones de comportamiento, uso y gestión de los edificios. 
\paragraph{inLab FIB}
InLab FIB se encargará del desarrollo de la herramienta.

El equipo dentro de inLab FIB esta formado por: dos desarrolladores (uno de los cuales es el autor de este proyecto) y dos \textit{project managers} (uno de los cuales será el director de este proyecto).

Como desarrolladores el objetivo de inLab FIB será finalizar el proyecto dentro de los términos establecidos, para concluirlo con éxito y conseguir la satisfacción final del cliente.

\paragraph{Director}
En el proyecto tiene el rol de \textit{project manager}.

Como miembro de inLab FIB su objetivo será el mismo que el del apartado anterior.
\paragraph{Autor}
Dentro del proyecto el autor tiene el rol de desarrollador.

De forma individual el objetivo del autor es entregar el proyecto dentro de los términos establecidos para poder finalizar el grado en ingeniería informática y como miembro de inLab FIB su objetivo será el mismo que el descrito anteriormente.
\paragraph{Usuario final}
El sistema esta dirigido a ellos. Los primeros usuarios finales serán una serie de edificios y usuarios para realizar pruebas de la comunidad de Madrid proporcionados por VIAS.

Posteriormente se espera que cualquier persona que quiera pueda entrar en el sistema.

El objetivo de los usuarios será tener una herramienta interactiva que a partir de los datos que ellos aporten, les dé una serie de consejos o recomendaciones para así mejorar el rendimiento energético de sus edificios.
\paragraph{Necada}
Necada~\cite{necada} será el desarrollador del sistema de cosimulación multiagente que será el núcleo del sistema.

El objetivo de Necada será obtener una herramienta interactiva para poder mostrar como funciona su sistema de simulación.
\paragraph{COOKIEBOX}
Cookiebox~\cite{cookiebox} es una empresa externa contratada para diseñar la parte de gamificación de la aplicación.

Su objetivo será entregar un diseño funcional dentro de los términos establecidos.
