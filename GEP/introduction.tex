\section{Introducción}
En la actualidad, el interés en temas de ahorro y eficiencia energética se encuentra en auge, esto es debido a la coyuntura económica que se esta atravesando (en particular el sector inmobiliario). A causa de esto, cada vez se apuesta más por propuestas de edificación y construcción sostenibles, para poder conseguir un ahorro tanto energético como económico. Actualmente las empresas que se dedican a la edificación y construcción sostenible, están optando por el desarrollo y la implementación de tecnologías de alto rendimiento.

\textbf{Lavola 1981,S.A.}~\cite{lavola} (en adelante LAVOLA) y \textbf{Vías y Construcciones, S.A.}~\cite{vias} (en adelante VIAS) son los clientes de este proyecto, los cuales se han percatado del gran potencial de ahorro energético presente en el parque de edificios existentes, es decir, la cantidad de energía que se esta malgastando en los edificios actuales y que debido muchas veces a la falta de información de los usuarios esto no se soluciona. Entonces es cuando se decide crear este proyecto que consistirá en el diseño y desarrollo de una herramienta destinada a mejorar la eficiencia energética de los edificios a través de la optimización del comportamiento de sus usuarios.

Para realizar esta herramienta se creará un sistema para que los usuarios puedan añadir y editar los datos de sus edificios y con estos datos, el sistema pueda ofrecer a los usuarios un seguido de recomendaciones para mejorar la eficiencia energética de sus edificios.

Este sistema estaría formado por un núcleo o cerebro de optimización de los parámetros de uso del edificio y por un sistema que servirá como enlace para interactuar con el usuario. El primero será una herramienta de cosimulación multiagente\footnote{NECADA\cite{necada} explicado con más detalle en la sección \ref{sec:necada}} y el segundo una aplicación gamificada.

Este proyecto de final de grado se centrará en el desarrollo de la aplicación gamificada que consistirá en el desarrollo de un portal web. Es decir, el objetivo de este proyecto será realizar una aplicación gamificada, que nutriéndose de los datos obtenidos de una herramienta de cosimulación multiagente, proporcione información útil para los usuarios y así mejorar la eficiencia energética de sus edificios.








