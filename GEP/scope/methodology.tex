\subsection{Metodología de trabajo}\label{sec:metodologia}
Este proyecto se trata de un consorcio formado por dos empresas ajenas a \textit{inLab FIB}, a causa de esto hay unas fechas programadas y las cuales se han de cumplir. Pese a esto, debido a la experiencia que \textit{inLab FIB} tiene, los buenos resultados que se han obtenido anteriormente y en vista que ciertos puntos de la aplicación no están definidos al cien por cien, se decide utilizar una metodología ágil, en concreto \textit{Scrum}.

El principio de \textit{Scrum} es trabajar mediante iteraciones, las cuales se denominan \textit{sprints}. Dentro de cada \textit{sprint} se definen una serie de tareas a realizar, las cuales el equipo se compromete a cumplir en el tiempo que duré el \textit{sprint}. La duración normal de un \textit{sprint} suele ser de una a cuatro semanas.

Estas iteraciones permiten obtener un \textit{feedback} continuo del proyecto e ir definiendo ciertas funcionalidades conforme este va avanzando. Esta metodología acepta y propicia la aparición de nuevos requisitos y la modificación de los ya existentes, para de esta manera obtener un producto que cumpla con todos los requisitos que quiera el cliente. 

Hay que puntualizar que esta metodología no se ha seguido al pie de la letra, en muchos casos se recomienda (por eso se llama metodología ágil) que cada equipo adapte este tipo de metodología como mejor se adapte a su forma de trabajo.

En este caso, las iteraciones tendrán una duración de cuatro semanas, al final de las cuales se realizará una reunión con los clientes para obtener el \textit{feedback} de estos y marcar las tareas a realizar durante las siguientes cuatro semanas.

\subsubsection{Herramientas de seguimiento}
Para realizar el control de versiones en este proyecto se ha utilizado \textit{Subversion~\cite{subversion}}\footnote{Herramienta de control de versiones \textit{open source} basada en repositorios}. Todos los miembros del equipo trabajan sobre el mismo repositorio, y es el propio \textit{Subversion} quien se encarga de gestionar los cambios realizados en los ficheros, de esta manera se obtiene un control de versiones al cual se puede acceder en cualquier momento y lugar.

Además de esto, se utilizará una herramienta \textit{online} llamada \textit{Trello~\cite{trello}} en la cual se podrá visualizar el trabajo que se esta llevando a cabo en todo momento. 

Finalmente, el seguimiento de todo el conjunto se llevará a cabo en las reuniones donde se podrá apreciar el trabajo realizado y si concuerda con las distintas herramientas utilizadas.

\subsubsection{Métodos de validación}
Los métodos de validación de este proyecto van atados a la planificación y a la metodología utilizada.

Dentro de la planificación~\footnote{Ver sección~\ref{sec:planning}} encontramos un seguido de tareas y subtareas. Estas son tareas generales que definen las partes del proyecto, pero para poder trabajar de una manera más ágil estas tareas se desgranaran durante los \textit{sprints} para tener tareas más asequibles dentro del tiempo que dure el \textit{sprint}.

Las tareas se validarán al final de cada \textit{sprint} durante las reuniones con los clientes y serán estos quienes se encargaran de validar el trabajo realizado, esta validación irá relacionada con la propia definición de la tarea en el \textit{sprint} anterior, con lo cual si el desarrollo cumple los requisitos que se encuentren en la tarea, esta será validada ya que eran los requisitos pactados dentro de ese \textit{sprint}.