\subsection{Alcance del Proyecto}\label{sec:alcance}
Como se ha dicho anteriormente este proyecto se centrara en la creación de un portal web para ayudar a los usuarios a mejorar la eficiencia energética de sus edificios, para así conseguir un ahorro tanto energético como económico. Para conseguir esto el portal web se nutrirá de la información obtenida por \textit{NECADA} y para atraer a los usuarios utilizará una estrategia de gamificación.

\subsubsection{Objetivos}
El objetivo principal del proyecto será la creación del portal, pero dentro de este objetivo general, se han definido una serie de objetivos específicos que van relacionados con las funcionalidades que se deberán implementar en el portal.
\paragraph{Gestión de usuarios}
Como ya se ha dicho, una parte muy importante dentro del proyecto, es la interacción con el usuario y que este facilite los datos de los edificios.

El sistema contará con dos tipos de usuario, por una parte los que nombraremos como \textit{Empresa} y por otra los \textit{Particulares}.

  Los usuario \textit{Empresa} en la primera prueba piloto solamente serán dos usuarios que representarán a VIAS y LAVOLA la particularidad de estos usuarios es que representan a empresas las cuales no querrán crear hogares por separado sino que buscarán crear edificios al completo. Pongamos como ejemplo una empresa constructora que crea un bloque de pisos nuevos decide dar de alta en el sistema todo el bloque que puede estar formado por veinte hogares.
  
  El resto de usuarios, \textit{Particulares}, representan a la gran mayoría de usuarios que se espera que utilicen el sistema, son usuarios los cuales tienen un hogar o varios y deciden registrar-lo en el sistema para poder obtener un mejor rendimiento energético.
  
  Para que el usuario sea capaz de todo esto, la aplicación permitirá registrarse, guardar los datos, y posteriormente mostrar la información que el usuario nos proporcione. Las primeras pruebas serán realizadas por un grupo de usuarios cerrado, pero una vez esta finalice se espera abrir el sistema para que todo aquel que quiera pueda utilizarlo.
\paragraph{Gestión de hogares}
Una vez un usuario se haya dado de alta en el sistema, para poder mejorar el rendimiento energético de sus edificios este tendrá que poder gestionarlos de alguna manera, para esto el sistema permitirá al usuario definir edificios (mediante un formulario del sistema) y posteriormente si desea modificarlos.

  Con estos datos y utilizando los datos obtenidos de \textit{NECADA} se espera calcular el consumo energético de los distintos edificios y que el usuario sea capaz de actualizarlos para así mejorar su eficiencia.
\paragraph{Sistema de recomendaciones}
Como sistema de recomendaciones se entiende que a partir de los datos que el usuario haya proporcionado y los datos obtenidos de NECADA, la aplicación sera capaz de proporcionar al usuario consejos y/o pautas para mejorar el rendimiento energético. Estas recomendaciones irán sujetas a la cantidad de datos que el usuario aporte, contra más datos aporte el usuario y más precisos sean de más ayuda serán estas indicaciones.
\paragraph{Sistema Gamificado}
Para atraer al mayor número posible de usuarios se desarrollará una estrategia de gamificación dentro del portal. Esta estrategia irá enfocada a los usuarios \textit{Particulares}, ya que este grupo de usuarios es más difícil que ingrese los datos de sus edificios de forma voluntaria y de esta forma poder ofrecerles unos objetivos.

\subsubsection{Obstáculos}
Antes de empezar un proyecto se han de detectar los posibles obstáculos que pueden aparecer durante el desarrollo. Para este proyecto, estos son los obstáculos que se han detectado:
\paragraph{Proyecto a definir}
El proyecto no está definido al cien por cien, esto es un gran riesgo a tener en cuenta, ya que ciertos casos de uso no están definidos des de el principio de este. Esto podría conllevar a retrasos ya que puede dar pie a que surjan nuevos casos de uso o se modifiquen los ya existentes.
Para que esto no ocurra el proyecto utilizará una metodología ágil\footnote{Explicada en mas detalle en la sección~\ref{sec:metodologia}} la cual esta pensada para proyectos cambiables como este.
\paragraph{Producto no deseado}
Al tratarse de un proyecto destinado a un cliente, podría darse el caso que el resultado final no fuera el que el cliente espera. Para evitar este obstáculo, se utilizará una metodología ágil que da un \textit{feedback} continuo a los clientes para que estos estén contentos con el producto final.
\paragraph{Control del tiempo}
Un riesgo que hay en casi todo proyecto es exceder las fechas limites. En este caso concreto, las fechas están definidas por un cliente además con lo que se tendrá que ser muy estricto con estas para que este esté satisfecho con el resultado final. Para evitar esto, se realizará un control exhaustivo para comprobar que se cumple la planificación establecida.
\paragraph{Datos externos}
La aplicación web se nutre de \textit{NECADA} por tanto si en algún momento, no podemos obtener los datos de este sistema, no podríamos seguir con ciertas funcionalidades del proyecto. El proyecto no depende completamente de \textit{NECADA} por tanto si esto ocurriera se podrían seguir realizando otras tareas hasta poder recibir los datos nuevamente.

