\section{Alcance del Proyecto}

El proyecto ha sido motivado dado el creciente interés en reducir el consumo energético de los edificios. Con este se busca ofrecer una herramienta interactiva capaz de incidir en los patrones de comportamiento, uso y gestión de los edificios por parte de los usuarios, de manera que se pueda maximizar la eficiencia y el ahorro energético durante la explotación y el mantenimiento de los mismos.

Este es fruto de un consorcio formado por dos empresas. Por una parte, Lavola 1981 S.A y por otra parte Vías y Construcciones, S.A. Además de este consorcio el proyecto cuenta con la colaboración del Laboratorio de Innovación e Investigación (inLab FIB), el cual se encargara de desarrollar dicha herramienta.

El proyecto se centrará en el desarrollo de una herramienta para ayudar a los usuarios a reducir el consumo energético de sus hogares a partir de los datos que estos nos facilite. Es decir, un sistema que en base a los datos que el usuario introduzca, este podrá ofrecerle una información dirigida a maximizar la eficiencia energetica.

La estructura del sistema estará formada primero por una herramienta de co-simulación multiagente\footnote{NECADA: Breu explicació de NECADA}, que se encargará de dar los parámetros para la optimización del edificio, y segundo una aplicación que será la que se encargará de interactuar con el usuario. 

Esta seria la estructura general del sistema, pero en este caso nos centraremos exclusivamente en la segunda parte del sistema, es decir, este proyecto se basa en la creación de la aplicación que interactuará con el usuario para que este le de información referente a sus edificios y a su vez ofrecer al usuario maneras de mejorar su rendimiento energético. Esta aplicación consistirá en un portal web que tendrá que tener las siguientes funcionalidades:
\begin{description}
  \item[Gestión de usuarios] \hfill \\
  Como ya hemos dicho una parte muy importante es la interacción con el usuario y que este facilite los datos de los edificios. Para esto la aplicación permitirá al usuario registrarse y guardar sus datos, para que posteriormente pueda disponer de la información que nos proporcione. Las primeras pruebas serán realizadas por un grupo de usuarios cerrado, pero posteriormente se espera abrir el sistema para que todo aquel que quiera pueda utilizarlo.
  \item[Gestión de hogares] \hfill \\
  Una vez un usuario se haya dado de alta en el sistema, para poder mejorar el rendimiento energético de sus edificios este tendrá que poder gestionarlos de alguna manera, para esto el sistema permitirá al usuario definir edificios (mediante un formulario del sistema) y posteriormente si desea modificarlos. Con estos datos se espera calcular el consumo energético de los distintos edificios y que el usuario sea capaz de actualizarlos para así mejorar su eficiencia.
  \item[Sistema de recomendaciones]\hfill \\
  Como sistema de recomendaciones entendemos que a partir de los datos que el usuario nos haya proporcionado y los datos obtenidos de la herramienta de co-simulación multiagente, la aplicación sera capaz de proporcionar al usuario consejos y/o pautas para mejorar el rendimiento energético. Estas recomendaciones irán sujetas a la cantidad de datos que el usuario aporte, contra más datos aporte el usuario y más precisos sean de más ayuda serán estas indicaciones. Sobretodo se tratan de recomendaciones que si el usuario no sigue no supondrán ningún tipo de mejora en el rendimiento.
\end{description}

Estas serian las tres funcionalidades principales de la aplicación y las que tendrá alcanzar este proyecto. Sin embargo, si los plazos lo permiten, se espera que una vez desarrolladas estas funcionalidades añadir una parte de gamificación a la aplicación. Esta parte gamificada seria de gran ayuda ya que como hemos dicho la aplicación se nutrirá de los datos que los usuarios proporcionen al sistema y con la gamificación se buscará incentivar al usuario a que facilite la mayor cantidad posible de datos.

Durante el desarrollo de la aplicación hay que tener en cuenta varios factores de riesgo, uno de los más importantes es que el proyecto no esta definido al cien por cien y esto puede conllevar a que surjan casos de uso nuevos o que los que ya existen sean ligeramente modificados, lo cual podría conllevar a retrasos en el proyecto. Otro factor es que como ya hemos dicho el sistema se divide en dos parte y nosotros solo nos ocupamos de la parte de la aplicación que interactúa con el usuario, al depender de un sistema externo esto podría ocasionar retrasos en ciertos puntos.

\subsection{Metodología de trabajo}

