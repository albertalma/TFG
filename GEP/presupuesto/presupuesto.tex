\subsection{Presupuesto}
\subsubsection{Identificación de los costes}
El presupuesto de un proyecto informático normalmente viene definido por una estimación en los costes de recursos humanos, hardware y software, además de estos costes se han de tener en cuenta los gastos generales del proyecto y los impuestos.

A partir de los diferentes costes asociados y utilizando como referencia el diagrama de Gannt de la figura \ref{fig:diagrama_gantt} se procede a calcular la estimación del presupuesto de este proyecto.

\subsubsection{Recursos humanos}
El coste principal de este proyecto es el de recursos humanos. Dentro del proyecto se cuenta con la participación de dos desarrolladores, \textit{Marc Juberó Silva} y el autor de este proyecto, nombrados como ''Desarrollador 1'' y "Desarrollador 2'' en las siguientes tablas. Además cuenta con la colaboración de dos \textit{Project Manager}, \textit{Jordi Montero García} y \textit{Marta Cuatrecasas Capdevila}, que aparecerán en las siguientes tablas como ''\textit{Project Manager} 1'' y ''\textit{Project Manager} 2''.  \\\\
Para calcular el coste total de recursos humanos es necesario saber el precio hora mínimo asociado a cada rol. En la tabla \ref{tab:salarioRol} se pueden ver los precios hora estimados.

\begin{table}[h]
\begin{center}
\begin{tabular}{|c|c|}
\hline
\rowcolor[HTML]{C0C0C0} 
\textbf{Rol}     & \textbf{Sueldo \euro /hora} \\ \hline
Project Manager  & 33.33                      \\ \hline
Desarrollador    & 7.5                      \\ \hline
\end{tabular}
\end{center}
\caption{Sueldo aproximado de cada rol involucrado en el proyecto. \label{tab:salarioRol}}
\end{table}

Puede sorprender que el sueldo del rol de desarrollador sea de 7.5 \euro/hora pero ambos desarrolladores son colaboradores en formación que cuentan con un convenio de Universidad-Empresa. Si se tratará de ingenieros informáticos el sueldo sería aproximadamente de unos 25 \euro/hora.

Una vez sabemos el precio hora de cada rol asociado al proyecto, mediante el diagrama de Gannt podemos obtener una estimación total del coste de recursos humanos, ahora mostraremos un resumen del coste de los diferentes roles asociado a las fases del proyecto.
\\\\
En la tabla \ref{tab:salarioDesarrolladores} se muestra el coste de los desarrolladores a partir de las distintas fases, el número de horas se ha calculado teniendo en cuenta que cada fase del proyecto dura de cuatro a ocho semanas y que un desarrollador realiza 20 horas semanales y otro 30 horas semanales.

\begin{table}[h]
\begin{center}
\begin{tabular}{ccccc}
\hline
\rowcolor[HTML]{C0C0C0} 
\multicolumn{1}{|c|}{\cellcolor[HTML]{C0C0C0}\textbf{Fase}} & \multicolumn{1}{p{3cm}|}{\cellcolor[HTML]{C0C0C0}\textbf{Horas Desarrollador 1}} & \multicolumn{1}{p{3cm}|}{\cellcolor[HTML]{C0C0C0}\textbf{Horas Desarrollador 2}} & \multicolumn{1}{c|}{\cellcolor[HTML]{C0C0C0}\textbf{Precio/Hora}} & \multicolumn{1}{c|}{\cellcolor[HTML]{C0C0C0}\textbf{Coste Fase}} \\ \hline
\multicolumn{1}{|c|}{Preparación Prévia}                    & \multicolumn{1}{c|}{80}                                                     & \multicolumn{1}{c|}{120}                                                    & \multicolumn{1}{c|}{7.5}                                          & \multicolumn{1}{c|}{1500}                                        \\ \hline
\multicolumn{1}{|c|}{Desarrollo Básico}                     & \multicolumn{1}{c|}{160}                                                    & \multicolumn{1}{c|}{240}                                                    & \multicolumn{1}{c|}{7.5}                                          & \multicolumn{1}{c|}{3000}                                        \\ \hline
\multicolumn{1}{|c|}{Desarrollo Avanzado}                   & \multicolumn{1}{c|}{160}                                                    & \multicolumn{1}{c|}{240}                                                    & \multicolumn{1}{c|}{7.5}                                          & \multicolumn{1}{c|}{3000}                                        \\ \hline
\multicolumn{1}{|c|}{Desarrollo Memoria}                    & \multicolumn{1}{c|}{0}                                                      & \multicolumn{1}{c|}{240}                                                    & \multicolumn{1}{c|}{7.5}                                          & \multicolumn{1}{c|}{1800}                                        \\ \hline
\rowcolor[HTML]{656565} 
{\color[HTML]{FFFFFF} \textbf{Total}}                       & {\color[HTML]{FFFFFF} \textbf{400}}                                         & {\color[HTML]{FFFFFF} \textbf{840}}                                         & {\color[HTML]{FFFFFF} \textbf{7.5}}                               & {\color[HTML]{FFFFFF} \textbf{9300}}                            
\end{tabular}
\end{center}
\caption{Coste de los desarrolladores a partir de las fases del proyecto. \label{tab:salarioDesarrolladores}}
\end{table}

Para calcular el coste de los \textit{Project Manager} también se tendrán en cuenta las horas dedicadas a cada fase. En este caso se estima que un \textit{Project Manager} dedica unas 6 horas semanales al proyecto ayudando en labores de desarrollo y unas 5 horas por cada \textit{sprint} teniendo en cuenta reuniones y gestión de estas.

\begin{table}[h]
\begin{center}
\begin{tabular}{ccccc}
\hline
\rowcolor[HTML]{C0C0C0} 
\multicolumn{1}{|c|}{\cellcolor[HTML]{C0C0C0}\textbf{Fase}} & \multicolumn{1}{p{3cm}|}{\cellcolor[HTML]{C0C0C0}\textbf{Horas Project Manager 1}} & \multicolumn{1}{p{3cm}|}{\cellcolor[HTML]{C0C0C0}\textbf{Horas Project Manager 2}} & \multicolumn{1}{c|}{\cellcolor[HTML]{C0C0C0}\textbf{Precio/Hora}} & \multicolumn{1}{c|}{\cellcolor[HTML]{C0C0C0}\textbf{Coste Fase}} \\ \hline
\multicolumn{1}{|c|}{Preparación Prévia}                    & \multicolumn{1}{c|}{24}                                                       & \multicolumn{1}{c|}{5}                                                        & \multicolumn{1}{c|}{33.33}                                        & \multicolumn{1}{c|}{966.57}                                      \\ \hline
\multicolumn{1}{|c|}{Desarrollo Básico}                     & \multicolumn{1}{c|}{48}                                                       & \multicolumn{1}{c|}{10}                                                       & \multicolumn{1}{c|}{33.33}                                        & \multicolumn{1}{c|}{1933.14}                                     \\ \hline
\multicolumn{1}{|c|}{Desarrollo Avanzado}                   & \multicolumn{1}{c|}{48}                                                       & \multicolumn{1}{c|}{10}                                                       & \multicolumn{1}{c|}{33.33}                                        & \multicolumn{1}{c|}{1933.14}                                     \\ \hline
\multicolumn{1}{|c|}{Desarrollo Memoria}                    & \multicolumn{1}{c|}{15}                                                       & \multicolumn{1}{c|}{5}                                                        & \multicolumn{1}{c|}{33.33}                                        & \multicolumn{1}{c|}{1800}                                        \\ \hline
\rowcolor[HTML]{656565} 
{\color[HTML]{FFFFFF} \textbf{Total}}                       & {\color[HTML]{FFFFFF} \textbf{135}}                                           & {\color[HTML]{FFFFFF} \textbf{30}}                                            & {\color[HTML]{FFFFFF} \textbf{33.33}}                             & {\color[HTML]{FFFFFF} \textbf{5499.45}}                         
\end{tabular}
\end{center}
\caption{Coste de los \textit{Project Manager} a partir de las fases del proyecto. \label{tab:salarioManager}}
\end{table}

Para acabar un resumen del coste total de recursos humanos se mostraría en la tabla \ref{tab:resumeHuman}, donde se muestra el número total de horas de cada rol y el coste total de cada uno de ellos.

\begin{table}[h]
\begin{center}
\begin{tabular}{cccc}
\hline
\rowcolor[HTML]{C0C0C0} 
\multicolumn{1}{|c|}{\cellcolor[HTML]{C0C0C0}\textbf{Rol}} & \multicolumn{1}{c|}{\cellcolor[HTML]{C0C0C0}\textbf{Horas Proyecto}} & \multicolumn{1}{c|}{\cellcolor[HTML]{C0C0C0}\textbf{Precio/Hora}} & \multicolumn{1}{c|}{\cellcolor[HTML]{C0C0C0}\textbf{Coste}} \\ \hline
\multicolumn{1}{|c|}{Desarrollador 1}                      & \multicolumn{1}{c|}{400}                                             & \multicolumn{1}{c|}{7.5}                                          & \multicolumn{1}{c|}{3000}                                   \\ \hline
\multicolumn{1}{|c|}{Desarrollador 2}                      & \multicolumn{1}{c|}{840}                                             & \multicolumn{1}{c|}{7.5}                                          & \multicolumn{1}{c|}{6300}                                   \\ \hline
\multicolumn{1}{|c|}{Project Manager 1}                    & \multicolumn{1}{c|}{135}                                             & \multicolumn{1}{c|}{33.33}                                        & \multicolumn{1}{c|}{4499.55}                                \\ \hline
\multicolumn{1}{|c|}{Project Manager 2}                    & \multicolumn{1}{c|}{30}                                              & \multicolumn{1}{c|}{33.33}                                        & \multicolumn{1}{c|}{999.9}                                  \\ \hline
\rowcolor[HTML]{656565} 
{\color[HTML]{FFFFFF} \textbf{Total}}                      & {\color[HTML]{FFFFFF} \textbf{1405}}                                 & {\color[HTML]{FFFFFF} \textbf{}}                                  & {\color[HTML]{FFFFFF} \textbf{14799.45}}                   
\end{tabular}
\end{center}
\caption{Coste total recursos humanos. \label{tab:resumeHuman}}
\end{table}

\subsubsection{Recursos materiales}
El coste de los recursos materiales los dividiremos en costes de hardware y costes de software. En la tabla \ref{tab:costSoftware} y en la tabla \ref{tab:costHardware} podemos ver los costes asociados a los distintos materiales.

\begin{table}[h]
\begin{center}
\begin{tabular}{|c|c|c|c|}
\hline
\rowcolor[HTML]{C0C0C0} 
\textbf{Hardware}   & \textbf{Unidades} & \textbf{Coste} & \cellcolor[HTML]{656565}{\color[HTML]{FFFFFF} Total} \\ \hline
Ordenador sobremesa & 2                 & 900            & \cellcolor[HTML]{656565}{\color[HTML]{FFFFFF} 1800}  \\ \hline
\end{tabular}
\end{center}
\caption{Material Hardware utilizado durante el proyecto. \label{tab:costHardware}}
\end{table}

\begin{table}[h]
\begin{center}
\begin{tabular}{ccc}
\hline
\rowcolor[HTML]{C0C0C0} 
\multicolumn{1}{|c|}{\cellcolor[HTML]{C0C0C0}\textbf{Software}} & \multicolumn{1}{c|}{\cellcolor[HTML]{C0C0C0}\textbf{Unidades}} & \multicolumn{1}{c|}{\cellcolor[HTML]{C0C0C0}\textbf{Coste}} \\ \hline
\multicolumn{1}{|c|}{Windows 8.1}                               & \multicolumn{1}{c|}{2}                                         & \multicolumn{1}{c|}{119}                                    \\ \hline
\multicolumn{1}{|c|}{Web Storm}                            & \multicolumn{1}{c|}{2}                                         & \multicolumn{1}{c|}{0}                                      \\ \hline
\multicolumn{1}{|c|}{Angular.js}                                & \multicolumn{1}{c|}{2}                                         & \multicolumn{1}{c|}{0}                                      \\ \hline
\multicolumn{1}{|c|}{Node.js}                                   & \multicolumn{1}{c|}{2}                                         & \multicolumn{1}{c|}{0}                                      \\ \hline
\multicolumn{1}{|c|}{Subversion}                                   & \multicolumn{1}{c|}{2}                                         & \multicolumn{1}{c|}{0}                                      \\ \hline
\multicolumn{1}{|c|}{Trello}                                   & \multicolumn{1}{c|}{2}                                         & \multicolumn{1}{c|}{0}                                      \\ \hline
\rowcolor[HTML]{656565} 
{\color[HTML]{FFFFFF} \textbf{Total}}                           & {\color[HTML]{FFFFFF} }                                        & {\color[HTML]{FFFFFF} 238}                                 
\end{tabular}
\end{center}
\caption{Material Software utilizado durante el proyecto. \label{tab:costSoftware}}
\end{table}

Estos son costes materiales, pero se ha de calcular la amortización dentro del proyecto, con lo cual tendremos en cuenta que el tiempo de vida del \textit{hardware} es de aproximadamente 4 años y el del \textit{software} de 3 años.

Primero de todo calcularemos la amortización del \textit{hardware} y del \textit{software} de la siguiente manera:
\begin{eqnarray} 
1800 \mbox{ \euro} / 4 \mbox{ años} = 450 \mbox{ \euro/año}
\end{eqnarray}

\begin{eqnarray} 
238 \mbox{ \euro} / 3 \mbox{ años} = 79.33 \mbox{ \euro/año}
\end{eqnarray}
Para obtener la amortización total multiplicamos la suma de la amortización del \textit{hardware} y del \textit{software} por la fracción de año que se dedicará al proyecto.

\begin{eqnarray} 
(450 \mbox{ \euro (Hardware)} + 79.33 \mbox{ \euro (Software)}) \cdot (7 \mbox{ meses} / 12 \mbox{ meses (1 año)}) = \textbf{308.78 \mbox{ \euro}}
\end{eqnarray}



















\subsubsection{Costes indirectos}
Para acabar de calcular el presupuesto del proyecto es necesario añadirle una serie de costes indirectos. En \textit{inLab FIB} se añade un 35\% a los proyectos para reflejar los costes indirectos asociados al \textit{inLab FIB} y a la \textit{UPC}.
\begin{eqnarray} 
(14799.45 \mbox{ Recursos Humanos} + 308.78 \mbox{ Recursos Materiales}) \cdot 1.35 = \textbf{20396.11 \mbox{ \euro} }
\end{eqnarray}
\subsubsection{Coste total}
Finalmente, para calcular el coste total del proyecto hay que añadir el impuesto del IVA (Impuesto de Valor Añadido) que actualmente es del 21\%, por tanto el coste total del proyecto sería de:
\begin{eqnarray} 
20396.11 \mbox{ \euro} \cdot 1.21 = \textbf{24679.3 \mbox{ \euro}}
\end{eqnarray}

\subsubsection{Control presupuesto}
El presupuesto de este proyecto va directamente relacionado con las horas dedicadas a su desarrollo. Por tanto, un aumento en la cantidad de horas del proyecto implicaría un aumento del coste de este.

Para evitar un aumento del presupuesto el equipo de desarrollo deberá restringirse a realizar solamente las tareas programadas. Por otra parte si se diera el caso que el cliente quisiera una nueva funcionalidad, se tendrá que prescindir de otra, de manera que esto no implique un aumento en el tiempo de desarrollo. 

Llegados a este punto tendrá que ser el cliente quien priorice que funcionalidades aportan más valor al proyecto y prefiere que se lleven a cabo dentro del tiempo establecido.
\subsubsection{Viabilidad económica}
El desarrollo de este proyecto esta financiado por los clientes que son las empresas \textit{VIAS} y \textit{LAVOLA}.

La finalidad de este proyecto es que tanto \textit{VIAS} como \textit{LAVOLA} al ser empresas especializadas en la construcción puedan utilizarlo de cara a futuros proyectos. Además se pretende comercializar de manera que otras empresas lo utilicen y buscar que el usuario de a pie también lo use.