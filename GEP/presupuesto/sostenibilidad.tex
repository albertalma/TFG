\subsection{Sostenibilidad y compromiso social}

El estudio de sostenibilidad se valorara des de tres puntos de vista: el económico, el social y el ambiental. Estos tres puntos se valorarán utilizando una puntuación siguiendo el método socrático, es decir, a partir de preguntas.

En la tabla \ref{tab:sostenibilidad} se muestra la matriz de sostenibilidad obtenida en este proyecto.

\begin{table}[h]
\begin{center}
\begin{tabular}{|c|c|c|c|c|}
\hline
\rowcolor[HTML]{9B9B9B} 
{\color[HTML]{FFFFFF} }               & {\color[HTML]{FFFFFF} Económica} & {\color[HTML]{FFFFFF} Social} & {\color[HTML]{FFFFFF} Ambiental} & \cellcolor[HTML]{343434}{\color[HTML]{FFFFFF} \textbf{Total}} \\ \hline
\cellcolor[HTML]{C0C0C0}Planificación & 9.1                              & 9.5                           & 9.3                              & \cellcolor[HTML]{343434}{\color[HTML]{FFFFFF} \textbf{9.3}}   \\ \hline
\end{tabular}
\end{center}
\caption{Matriz de sostenibilidad del proyecto. \label{tab:sostenibilidad}}
\end{table}

La justificación de la matriz viene explicada en los siguientes apartados.

\subsubsection{Dimensión económica}
Para calcular la puntuación de la dimensión económica se tendrán en cuenta las siguientes afirmaciones:

\begin{itemize}
  \item Tal y como se refleja en el presupuesto se ha realizado una evaluación exhaustiva de los distintos costes dentro del proyecto.
  \item Una vez se entregue el proyecto finalizado, este será responsabilidad del cliente. Si este quisiera un mantenimiento se debería realizar otro presupuesto como si de otro proyecto se tratase.
  \item El coste del proyecto es muy ajustado, teniendo en cuenta que los desarrolladores cuentan con un sueldo de convenio Universidad-Empresa, es muy probable que si se hubiera realizado por ingenieros informáticos el precio hubiera sido mucho mayor. Considerando esto, el coste del proyecto lo haría viable si tuviera que ser competitivo.
  \item En menos tiempo es posible que si, ya que los desarrolladores no son expertos en la materia. Si se hubiera contratado personal especializado, se podría acortar el proyecto, pero esto también conllevaría un coste mayor.
  \item Las tareas como se puede ver en el diagrama de gannt de la figura \ref{fig:diagrama_gantt} tienen distintas duraciones y las que se estiman más complicadas o de mayor importancia tiene asociado un tiempo mayor.
  \item El proyecto cuenta con una subvención del Centro para el Desarrollo Tecnológico Industrial (CDTI), en la modalidad de Proyectos de I'D en Cooperación Nacional.
\end{itemize}

Teniendo en cuenta estas afirmaciones la puntuación de la dimensión económica seria un 9.1 .

\subsubsection{Dimensión social}
Para calcular la puntuación de la dimensión social se tendrán en cuenta las siguientes afirmaciones:

\begin{itemize}
  \item Actualmente el país donde se desarrolla este proyecto esta pasando por una coyuntura económica bastante mala y muchos sectores, sobretodo el de la construcción, se están viendo afectados por esto.
  \item Gracias a este proyecto se espera mejorar la eficiencia energética de los usuarios, lo que conllevará a un ahorro económico.
  \item Existe una gran capacidad de ahorro energético dentro de los edificios ya existentes, por tanto un producto como este puede favorecer a mejorar la eficiencia de estos.
  \item Al conseguir un ahorro energético, los usuarios a su vez conseguirán un ahorro económico.
  \item No solo se espera conseguir un ahorro económico para el usuario, si no que además se busca concienciarlo provocando un cambio en el comportamiento de este.
  \item Ningún colectivo debería verse afectado de forma negativa por este proyecto.
\end{itemize}

Teniendo en cuenta estas afirmaciones la puntuación de la dimensión social seria un 9.5 .
\subsubsection{Dimensión ambiental}

Para calcular la puntuación de la dimensión ambiental se tendrán en cuenta las siguientes afirmaciones:

\begin{itemize}
  \item El único recurso que se necesitara durante el proyecto será electricidad, el cual no afecta de forma directa al medio ambiente. Si no que dependerá de como se haya producido (energías renovables o no renovables).
  \item Si no se realizará este TFG el impacto sería prácticamente el mismo, ya que por el hecho de realizarlo la única diferencia podría ser si se tuviera que imprimir la memoria, en caso contrario conllevaría el mismo gasto.
  \item No se puede reutilizar nada de otros proyectos.
  \item El proyecto no producirá contaminación de forma directa, como mucho se puede contemplar el gasto eléctrico.
  \item El proyecto no requiere de ningún tipo de material manufacturado.
  \item Con la implantación de este proyecto se espera mejorar la eficiencia energética de los edificios y concienciar a los usuarios, con lo cual uno de sus posibles resultados será la disminución de la huella ecológica.
\end{itemize}

Teniendo en cuenta estas afirmaciones la puntuación de la dimensión ambiental seria un 9.3 .
