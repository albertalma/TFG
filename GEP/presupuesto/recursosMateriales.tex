\subsubsection{Recursos materiales}
El coste de los recursos materiales los dividiremos en costes de hardware y costes de software. En la tabla \ref{tab:costSoftware} y en la tabla \ref{tab:costHardware} podemos ver los costes asociados a los distintos materiales.

\begin{table}[h]
\begin{center}
\begin{tabular}{|c|c|c|c|}
\hline
\rowcolor[HTML]{C0C0C0} 
\textbf{Hardware}   & \textbf{Unidades} & \textbf{Coste} & \cellcolor[HTML]{656565}{\color[HTML]{FFFFFF} Total} \\ \hline
Ordenador sobremesa & 2                 & 900            & \cellcolor[HTML]{656565}{\color[HTML]{FFFFFF} 1800}  \\ \hline
\end{tabular}
\end{center}
\caption{Material Hardware utilizado durante el proyecto. \label{tab:costHardware}}
\end{table}

\begin{table}[h]
\begin{center}
\begin{tabular}{ccc}
\hline
\rowcolor[HTML]{C0C0C0} 
\multicolumn{1}{|c|}{\cellcolor[HTML]{C0C0C0}\textbf{Software}} & \multicolumn{1}{c|}{\cellcolor[HTML]{C0C0C0}\textbf{Unidades}} & \multicolumn{1}{c|}{\cellcolor[HTML]{C0C0C0}\textbf{Coste}} \\ \hline
\multicolumn{1}{|c|}{Windows 8.1}                               & \multicolumn{1}{c|}{2}                                         & \multicolumn{1}{c|}{119}                                    \\ \hline
\multicolumn{1}{|c|}{Web Storm}                            & \multicolumn{1}{c|}{2}                                         & \multicolumn{1}{c|}{0}                                      \\ \hline
\multicolumn{1}{|c|}{Angular.js}                                & \multicolumn{1}{c|}{2}                                         & \multicolumn{1}{c|}{0}                                      \\ \hline
\multicolumn{1}{|c|}{Node.js}                                   & \multicolumn{1}{c|}{2}                                         & \multicolumn{1}{c|}{0}                                      \\ \hline
\multicolumn{1}{|c|}{Subversion}                                   & \multicolumn{1}{c|}{2}                                         & \multicolumn{1}{c|}{0}                                      \\ \hline
\multicolumn{1}{|c|}{Trello}                                   & \multicolumn{1}{c|}{2}                                         & \multicolumn{1}{c|}{0}                                      \\ \hline
\rowcolor[HTML]{656565} 
{\color[HTML]{FFFFFF} \textbf{Total}}                           & {\color[HTML]{FFFFFF} }                                        & {\color[HTML]{FFFFFF} 238}                                 
\end{tabular}
\end{center}
\caption{Material Software utilizado durante el proyecto. \label{tab:costSoftware}}
\end{table}

Estos son costes materiales, pero se ha de calcular la amortización dentro del proyecto, con lo cual tendremos en cuenta que el tiempo de vida del \textit{hardware} es de aproximadamente 4 años y el del \textit{software} de 3 años.

Primero de todo calcularemos la amortización del \textit{hardware} y del \textit{software} de la siguiente manera:
\begin{eqnarray} 
1800 \mbox{ \euro} / 4 \mbox{ años} = 450 \mbox{ \euro/año}
\end{eqnarray}

\begin{eqnarray} 
238 \mbox{ \euro} / 3 \mbox{ años} = 79.33 \mbox{ \euro/año}
\end{eqnarray}
Para obtener la amortización total multiplicamos la suma de la amortización del \textit{hardware} y del \textit{software} por la fracción de año que se dedicará al proyecto.

\begin{eqnarray} 
(450 \mbox{ \euro (Hardware)} + 79.33 \mbox{ \euro (Software)}) \cdot (7 \mbox{ meses} / 12 \mbox{ meses (1 año)}) = \textbf{308.78 \mbox{ \euro}}
\end{eqnarray}


















